Social media cyberbullying continues to expand so researchers have developed detection systems which apply ML and DL technology to analyze linguistic and multichannel inputs for abusive content. The combination of network processing techniques with neural network technology enables identification of verbal along with coded and visual harassment. Preventing detection of covert offensive language proves difficult due to the presence of slang terms and abbreviations together with emoticons and memes. The contextual abilities of BERT-like models excel but they face difficulties when processing casual dialogue. Word innovations tend to evolve more quickly than natural language processing capabilities do. The available models exclusively target Twitter and Facebook while lacking flexibility in their design. The development of powerful systems that maintain stable performance dynamics across distinct user engagements depends on applying transfer learning to multi-platform database inputs.



% A social media platform threat grows steadily due to the expansion of cyberbullying incidents. Research showed how ML and DL algorithms operate in cyberbullying detection systems by using linguistic and multichannel inputs to identify abusive content. Traditional network processing when integrated with neural networks enables the monitoring of harassment patterns which include verbal as well as coded and visual content.

% Main obstacles persist in detecting hidden offensive language from non-standard expressions which include slang and shortened text combined with emoticons and meme-based communications. Recent word innovations appear faster than regular NLP models can track them. The current version of BERT-like models shows advanced contextual analysis but performs poorly in processing informal speech patterns. For effective detection the present frameworks need specialized pre-processing methods or new architectural frameworks.

% Current detection models focus exclusively on specific platforms such as Twitter and Facebook while they struggle to adjust their capabilities across various platforms that each have their own distinct user actions. To achieve accurate detection capabilities while maintaining model performance researchers must create adaptable models using transfer learning methods on diverse multi-platform data collections.
% Limited evidence exists regarding how cyberbullying impacts individuals psychologically and neurobiologically through depression and anxiety, together with alterations in neural activity. Early intervention strength increases alongside the implementation of behavioral analysis with sentiment metrics and psychological data that provides personalized assistance for both victims and aggressors. The deployment of smart detection systems creates multiple ethical problems along with privacy violations. System performance requires strict privacy safeguards along with fairness audits and explainable AI (XAI) to prevent flagging biases related to gender, race or language in order to maintain responsible, effective system performance.

% The research field dedicated to studying psychological and neurobiological impacts of cyberbullying through anxiety development and depression increase and neural system modification shows inadequate investigation. The integration of behavioral analysis with sentiment metrics alongside psychological assessment data enhances system detection capabilities so that both victim and aggressor receive early warning and tailored assistance.

% Smart detection systems create ethical problems alongside privacy problems due to their deployment. Clear processes alongside strict protection of privacy standards remain vital since algorithmic biases cause improper content flags that stem from gender, racial or language-based classification errors. AI system performance improves with responsible practices that include both fairness audits and explainable AI (XAI).

% The creation of an ideal cyberbullying detection system needs support at multiple levels from professionals who demonstrate expertise in computation as well as psychology and linguistics and follow ethical standards. Danger detection using deep learning models performs real-time cyberbullying surveillance of diverse content but needs additional development. The future development of cyberbullying prevention systems needs psychological indicators and ethical assessments combined with technician-sociologist-policymaker-psychiatrist collaboration to achieve secure and ethical mechanisms that prevent cyberbullying.

% The following phase of model development should concentrate on building systems capable of analyzing formal along with informal abusive content by incorporating psychological indicators for enhanced detection precision. The objective of achieving ethical analysis and decreasing detection system biases requires technological specialists to partner with psychiatrists sociologists and policymakers.

% The battle to eliminate cyberbullying involves more than implementing technical solutions. In order to achieve secure digital environments the world needs continuous creativity alongside ethical practices and unified goals.

% A subsequent development stage must include psychological markers to improve the identification of abusive content in all forms. The elimination of biases and ethical analysis needs technicians to work together with psychiatrists sociologists and policymakers. Secure digital environments will be achieved through continuous creative measures and unified efforts and ethical practices against cyberbullying.

