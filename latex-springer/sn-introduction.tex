Extensive social communication opportunities are currently delivered through digital platforms via social media as one of the dominant trends of the time. The digital devices uniting Facebook, Instagram, TikTok and Twitter users across four billion five hundred million worldwide individuals produce cyberbullying as a destructive result of electronic social interactions that affects young people especially.\cite{bib7}\cite{bib8} Victims of cyberbullying encounter concentrated digital attacks through practices like social isolation followed by intentional harassment and identity deception.  Virtual means provide sustenance to those who commit harmful actions because of the combination between geographical distance and anonymous interactions which enables their harmful behavior to persist without direct consequences.

Research indicates that between 15 and 20 percent of kids experience some form of online harassment, which is a startlingly high prevalence of cyberbullying. \cite{bib7}  The victim may experience severe emotional and psychological effects, often including elevated anxiety, sadness, and, in severe situations, suicide thoughts. \cite{bib21}  Beyond issues of personal health, the pervasiveness of cyberbullying ruins peer relationships\cite{bib1}, alters social dynamics, and fosters a toxic environment that can impede academic achievement and personal development.  The necessity of effective detection and intervention systems becomes crucial because of their vital role.

Manual reporting and keyword scanning systems used to detect cyberbullying prove inadequate because they cannot identify advanced forms of cyber harassment documented in references 9 and 10. Traditional detection systems fail at interpreting social media communication since they lack the capability to decode the hidden abusive messages that use coded expressions combined with slang and emoji language\cite{bib8}. The detection of cyberbullying becomes difficult because victims do not report incidents and detection systems cannot view the lowest possible intensity of cyberbullying scenarios. The victims remain unprotected when no support measures are in place. Current detection systems fail to prevent cyberbullying completely which proves the urgent necessity for improved systems that can recognize abuse events. 

The research develops a detection system with advanced ML and DL methods to address this pivotal problem. A fast-acting, aggressive social media analytics system has been developed through the combination of text and image processing approaches. This system provides real-time reporting along with alert capabilities to enable both victims and moderators to swiftly act upon cyberbullying incidents, thereby safeguarding people from such damage.

This investigation produces results that surpass detection methods by advancing the security of online environments for multiple user groups. Detection strategies aim to develop equal offerings in digital space while recognizing both data-management moral concerns and changing online interaction dynamics. Collaboration among scientific researchers, practitioners, and social media companies is necessary to stop cyberbullying, as this activity represents an essential social responsibility. Joint efforts in fighting online harassment produce optimal results when communities unite to develop supportive spaces in digital environments.